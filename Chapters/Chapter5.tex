% Chapter Template

\chapter{Conclusiones} % Main chapter title

\label{Chapter5} % Change X to a consecutive number; for referencing this chapter elsewhere, use \ref{ChapterX}


%----------------------------------------------------------------------------------------
%	Chapter 5
%----------------------------------------------------------------------------------------

El último capítulo resume los resultados alcanzados en este trabajo, el grado de cumplimiento de los requerimientos y plantea las mejoras necesarias en etapas futuras.

\section{Conclusiones generales}

En líneas generales, a pesar de los contratiempos encontrados durante el desarrollo, el trabajo se llevó a cabo con éxito ya que:  

\begin{itemize}
\item Se obtuvo un prototipo validado del hardware, lo que permite volver a usar el mismo diseño y componentes en el producto final.
\item Se desarrolló una primera versión validada del firmware.
\item Se logró interiorizarse en la tecnología de los EDFA y su funcionamiento.
\item Como se explica en la sección \ref{sec:dispImp}, el dispositivo desarrollado no es el producto final si no una primera versión que funciona como prueba de concepto y simulador del comportamiento del amplificador. Aún así, logra cumplir con los requerimientos principales establecidos por la empresa, lo que permite darlo por finalizado y pasar a la etapa de desarrollo del producto final.
\item Se hizo uso de muchas herramientas, conceptos y metodologías incorporadas durante la cursada de la especialización, dando como resultado un trabajo profesional y facilitando su desarrollo.
\end{itemize}

\section{Trabajo futuro}

Con el objetivo de contar con un producto final apto para uso en tareas de integración e investigación y desarrollo, se planea avanzar en varios aspectos. Entre estos se encuentran:

\begin{itemize}
\item El rediseño y la fabricación de la versión final del PCB del tester, corrigiendo los errores de diseño encontrados en el prototipo. También se deberá incluir el microcontrolador de la placa NUCLEO-144 o un equivalente junto con su interfaz de programación. Todo esto se debe hacer teniendo en cuenta su factor de forma ya que se deberá conectar a un EDFA y montarse sobre una carcasa. 
\item Mejorar el firmware añadiendo más funcionalidades y robustecerlo corrigiendo errores.
\item Una vez rediseñado el PCB, volver a validar el hardware y firmware utilizando un EDFA provisto por la empresa.
\item Realizar las modificaciones necesarias en el firmware para lograr compatibilidad con los \textit{scripts} de Python que la empresa utiliza para ejecutar pruebas de aceptación y funcionamiento automatizadas.
\end{itemize}
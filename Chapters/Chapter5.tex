% Chapter Template

\chapter{Conclusiones} % Main chapter title

\label{Chapter5} % Change X to a consecutive number; for referencing this chapter elsewhere, use \ref{ChapterX}


%----------------------------------------------------------------------------------------
%	Chapter 5
%----------------------------------------------------------------------------------------

El último capítulo resume los resultados alcanzados en este trabajo, el grado de cumplimiento de los requerimientos y plantea las mejoras necesarias en etapas futuras.

\section{Conclusiones generales}

En lineas generales el desarrollo del proyecto se llevó a cabo con éxito ya que, a pesar de los contratiempos encontrados, se logró:  

\begin{itemize}
\item Obtener un prototipo validado del hardware
\item Desarrollar una primera versión validada del firmware 
\item Interiorizarse en la tecnología de los EDFA y su funcionamiento
\item Como se explica en la sección \ref{sec:dispImp}, el dispositivo desarrollado no es el producto final si no una primera versión que funciona como prueba de concepto y simulador del comportamiento del amplificador. Aún así, logra cumplir con los requerimientos principales establecidos por la empresa.
\end{itemize}

\section{Trabajo futuro}

Con el objetivo de contar con un producto final apto para uso en tareas de integración e investigación y desarrollo,  se planea avanzar en varios aspectos. Entre estos se encuentran:

\begin{itemize}
\item Rediseñar el PCB a la versión final corrigiendo los errores encontrados y aplicando las modificaciones necesarias para poder conectarse a un EDFA y montarse sobre la carcasa. También se deberá incluir el microcontrolador de la placa Nucleo junto con su interfaz de programación
\item Robustecer el firmware añadiendo mas funcionalidades y corrigiendo errores
\item Una vez rediseñado el PCB, volver a validar el hardware y firmware utilizando un EDFA provisto por la empresa
\end{itemize}
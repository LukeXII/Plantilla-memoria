\chapter{Diseño e implementación} % Main chapter title

\label{Chapter3} % Change X to a consecutive number; for referencing this chapter elsewhere, use \ref{ChapterX}

\definecolor{mygreen}{rgb}{0,0.6,0}
\definecolor{mygray}{rgb}{0.5,0.5,0.5}
\definecolor{mymauve}{rgb}{0.58,0,0.82}

%%%%%%%%%%%%%%%%%%%%%%%%%%%%%%%%%%%%%%%%%%%%%%%%%%%%%%%%%%%%%%%%%%%%%%%%%%%%%
% parámetros para configurar el formato del código en los entornos lstlisting
%%%%%%%%%%%%%%%%%%%%%%%%%%%%%%%%%%%%%%%%%%%%%%%%%%%%%%%%%%%%%%%%%%%%%%%%%%%%%
\lstset{ %
  backgroundcolor=\color{white},   % choose the background color; you must add \usepackage{color} or \usepackage{xcolor}
  basicstyle=\footnotesize,        % the size of the fonts that are used for the code
  breakatwhitespace=false,         % sets if automatic breaks should only happen at whitespace
  breaklines=true,                 % sets automatic line breaking
  captionpos=b,                    % sets the caption-position to bottom
  commentstyle=\color{mygreen},    % comment style
  deletekeywords={...},            % if you want to delete keywords from the given language
  %escapeinside={\%*}{*)},          % if you want to add LaTeX within your code
  %extendedchars=true,              % lets you use non-ASCII characters; for 8-bits encodings only, does not work with UTF-8
  %frame=single,	                % adds a frame around the code
  keepspaces=true,                 % keeps spaces in text, useful for keeping indentation of code (possibly needs columns=flexible)
  keywordstyle=\color{blue},       % keyword style
  language=[ANSI]C,                % the language of the code
  %otherkeywords={*,...},           % if you want to add more keywords to the set
  numbers=left,                    % where to put the line-numbers; possible values are (none, left, right)
  numbersep=5pt,                   % how far the line-numbers are from the code
  numberstyle=\tiny\color{mygray}, % the style that is used for the line-numbers
  rulecolor=\color{black},         % if not set, the frame-color may be changed on line-breaks within not-black text (e.g. comments (green here))
  showspaces=false,                % show spaces everywhere adding particular underscores; it overrides 'showstringspaces'
  showstringspaces=false,          % underline spaces within strings only
  showtabs=false,                  % show tabs within strings adding particular underscores
  stepnumber=1,                    % the step between two line-numbers. If it's 1, each line will be numbered
  stringstyle=\color{mymauve},     % string literal style
  tabsize=2,	                   % sets default tabsize to 2 spaces
  title=\lstname,                  % show the filename of files included with \lstinputlisting; also try caption instead of title
  morecomment=[s]{/*}{*/}
}


%----------------------------------------------------------------------------------------
%	Chapter 3
%----------------------------------------------------------------------------------------

En este capítulo se realiza una explicación detallada del diseño del hardware, como se utilizó cada componente principal y su conexión con el microcontrolador. Asimismo, se especifica la estructura general del firmware en el marco de FreeRTOS y las principales rutinas de bajo nivel.

\section{Dispositivo implementado}

Es importante aclarar que el dispositivo que se desarrolló e integró como solución a lo planteado en el capítulo 1 no es el producto final destinado a usarse. La placa construida se puede interpretar como un prototipo de desarrollo o una versión de ingeniería ya que tiene dos objetivos: verificar el funcionamiento del hardware diseñado y desarrollar el firmware que se ejecutará en el microcontrolador. 

La versión armada para este proyecto, además de incluir los mismos componentes que se pretenden usar en el producto final, cuenta con todo el hardware necesario para simular la interfaz electrónica del modelo del amplificador óptico. Esto permite generar todas las señales que se encuentran en el conector, tanto las analógicas como las digitales.

Una vez alcanzados estos objetivos, la siguiente etapa consiste en desarrollar la placa del producto final, con el factor de forma adecuado para integrar y montar todos los componentes en una carcasa transportable. 

El PCB fue enviado a fabricar a la empresa china JLC y consta de 4 capas: dos planos internos de referencia conectados a 3.3 V y GND, y dos capas de señal (\textit{Top} y \textit{Bottom}). En la figura \ref{fig:placa1} y \ref{fig:placa2} se pueden ver dos fotos del PCB ensamblado sin el display ni la placa Nucleo conectados (lado superior y lado inferior respectivamente). Por último, en la figura \ref{fig:placa3} se puede ver la placa con ambos módulos conectados.

\begin{figure}[H]
     \centering
     \begin{subfigure}{\textwidth}
         \centering
         \includegraphics[width=.65\textwidth]{./Figures/placa2.jpg}
         \caption{Lado superior del PCB.}
         \label{fig:placa1}
     \end{subfigure}
     \begin{subfigure}{\textwidth}
         \centering
         \includegraphics[width=.65\textwidth]{./Figures/placa3.jpg}
         \caption{Lado inferior del PCB.}
         \label{fig:placa2}
     \end{subfigure}
     \begin{subfigure}{\textwidth}
         \centering
         \includegraphics[width=.65\textwidth]{./Figures/placa1.jpg}
         \caption{Placa ensamblada.}
         \label{fig:placa3}
     \end{subfigure}
        \caption{Placa fabricada para el proyecto}
        \label{fig:three graphs}
\end{figure}

\section{Arquitectura de hardware}

La arquitectura general de la placa fue diseñada para tener como unidad de procesamiento principal al microcontrolador STM32 de la placa Nucleo y a la pantalla LCD como periférico de entrada y salida del usuario.

\subsection{Diagrama del dispositivo}

La figura \ref{fig:estrucInt} muestra un diagrama detallado y completo de todo el hardware de la placa y el conexionado entre sus distintas partes. También incluye los periféricos utilizados en el microcontrolador.

%\begin{figure}[H]
%\centering
%\includegraphics[width=0.85\textwidth]{./Figures/.png}
%\caption{Estructura interna del sistema}
%\label{fig:estrucInt}
%\end{figure}

\subsection{Medición de corriente del amplificador}

El principio de funcionamiento del monitor de corriente se basa en un resistor \textit{shunt}, el cual se coloca sobre la alimentación, en serie con la línea sobre la cual se desea conocer el consumo de corriente. Para que este resistor no disipe mucha potencia y disminuya la tensión de la línea, su valor de resistencia es de 10 \si{m\ohm}). Mientras más chico sea su valor mas despreciables son los efectos negativos que introducirá.

Sobre este resistor cae una tensión proporcional a la corriente consumida por el EDFA. Como el valor máximo de corriente consumida por el amplificador se encuentra cerca de los 2.5 A entonces sobre el resistor caerán como máximo 25 mV, lo que frente a los 5 V de la alimentación no representa un valor significativo. Luego de sensarla, el chip amplifica 100 veces ésta tensión diferencial y la conecta a uno de sus pines.

Al ser éste un valor analógico, para poder ser interpretado se lo debe convertir a digital. Para esto se usó el ADC incorporado en el microcontrolador, conectando a uno de sus pines la señal de salida del monitor de corriente. La tensión máxima que alcanzará el pin será de aproximadamente 2.5 V, lo que se encuentra dentro de la tensión de referencia del ADC. Un esquema de esta conexión se puede ver en la figura \ref{fig:funcMonitor}.

\begin{figure}[H]
\centering
\includegraphics[width=0.75\textwidth]{./Figures/func_monitor.png}
\caption{Esquema del monitor de corriente}
\label{fig:funcMonitor}
\end{figure}

Por otro lado, el INA301A3 cuenta con un pin de salida denominado ALERT que se activa cuando la corriente medida alcanza cierto valor, funcionando a modo de alarma. Este nivel se determina conectando un resistor de cierto valor en el pin LIMIT.

\subsection{Relé de alimentación}

La línea de 5 V de alimentación del EDFA es controlada mediante la activación de un relé (código APAN3103), lo que significa que la conexión y desconexión es mecánica. Esto se decidió implementarlo así debido a los lineamientos de protecciones necesarios para el hardware de vuelo.

Como el relé necesita de 36.7 mA para activarse, no se puede conectar directamente a un pin de salida del microcontrolador porque no podrá suministrar la corriente necesaria. Por ésta razón la activación se realiza mediante un transistor (código MMBT2222), funcionando como una llave (en modo saturación o corte). De ésta forma la corriente para la activación de la bobina del relé la proveerá directamente la fuente de alimentación de 3.3 V.

\subsection{Medición de tensión del amplificador}

Hacer una medición de la tensión de la línea es mucho mas sencillo que hacer una medición de la corriente, ya que esta se puede conectar directamente a un ADC si previamente se la atenúa o amplifica (dependiendo del caso).

En este caso, como la tensión de alimentación del EDFA es 5 V y la tensión de referencia del ADC del microcontrolador es 3.3 V, se la debe atenuar de forma que esta se encuentre dentro del fondo de escala del ADC.

Considerando que se desea medir una sobretensión de por lo menos 1 V en la alimentación, se optó por atenuar la tensión del EDFA a la mitad, resultando en un fondo de escala de 6.6 V.

Esta técnica permite medir tensiones mayores a la de referencia del ADC pero trae como desventaja que se pierde precisión debido al ruido y la tolerancia de los resistores.

En la figura \ref{fig:monTension} se puede ver el circuito implementado.

\begin{figure}[H]
\centering
\includegraphics[width=0.5\textwidth]{./Figures/mon_tension.png}
\caption{Circuito del monitor de corriente}
\label{fig:monTension}
\end{figure}

El capacitor C6 forma un filtro RC para eliminar el ruido de alta frecuencia que pueda tener la línea.

\subsection{Conexión a pantalla LCD}

Como se mencionó en la subsección \ref{sec:pantLCD}, la pantalla LCD tiene dos circuitos integrados con SPI para el control de la pantalla y para el control de la función táctil. Si bien ambas interfaces se podrían haber conectado al mismo periférico SPI del microcontrolador en modo multi-esclavo de la forma en que se muestra en la figura \ref{fig:multiSlave}, se decidió usar uno para cada uno de forma individual. La principal razón de ésto es porque facilitaba la implementación del firmware en gran medida, ya que permitía usar ambos controladores de forma independiente en distintas partes del programa sin riesgo de colisión. De lo contrario, podría suceder que ambos drivers intenten acceder al periférico al mismo tiempo.

El controlador de la función táctil tiene, además de las señales del bus SPI, una señal denominada IRQ que se activa cuando se detecta que la pantalla está siendo tocada. Ésta señal sirve para ser usada como interrupción en el programa del microcontrolador.

Por otro lado, el controlador del display también tiene señales con funciones especiales. Éstas son:

\begin{itemize}
\item RESET: resetea el chip del controlador y por lo tanto toda la configuración del display y la información de cada píxel
\item DC/RS: indica al controlador si la información que se está enviando por el SPI es un dato o un registro, dependiendo del nivel de la señal
\item LED: permite controlar la intensidad de luz de la pantalla (denominada \textit{backlight})
\end{itemize}

Teniendo en cuenta ésto la conexión entre el módulo del display y el microcontrolador resulta como se ve en la figura \ref{fig:conexionLCD}.

%\begin{figure}[H]
%\centering
%\includegraphics[width=0.5\textwidth]{./Figures/diagFSM.jpg}
%\caption{Máquina de estados.}
%\label{fig:conexionLCD}
%\end{figure}

\subsection{Entradas y salidas digitales}

En la tabla \ref{tab:señalesConector} se detallan las señales de entrada y salida digitales del puerto del EDFA. Las salidas son señales de alarmas y las de entrada, de control.

Éstas señales se podrían conectar directamente a pines digitales del microcontrolador ya que son de tipo TTL 3.3 V, pero se decidió poner antes de este un \textit{buffer} de tres estados (código SN74LVC2244A). Los objetivos de ésta decisión son tres:

\begin{itemize}
\item Generar niveles de tensión mejor definidos. Ésto filtra ruido y rebote en las señales y facilita la detección en ambos extremos (EDFA y microcontrolador)
\item Proveer a la electrónica interna del EDFA de protección contra descargas electrostáticas (ESD)
\item Suministrar la corriente necesaria a las señales de entrada del amplificador y así evitar que ésta deba ser proporcionada por el microcontrolador
\end{itemize}

Para poder simular la activación de las alarmas en la placa se colocaron pulsadores con testigo lumínico (código TL1265YQSCLR) en el extremo donde estaría el EDFA, con el propósito de activarlos manualmente durante el testeo del firmware.

Para las señales de control se pusieron LEDs que indican el estado de activación de las mismas (código AA3528LVBS/D).

\subsection{Indicadores de potencia óptica}

Las señales IN\_POW y OUT\_POW son analógicas y pueden tomar cualquier valor en el rango de 0 V a 3.3 V. Éste valor es proporcional a la potencia óptica medida por los detectores de entrada y salida del amplificador, el cual corresponde al rango de -10 dBm a 10 dBm para el de entrada y 5 dBm a 37 dBm para el de salida.

Para poder simular éstas señales se utilizaron dos resistores variables (código PTV09A-4020U-B103) que permiten variar de forma manual el valor de tensión entre 0 V y 3.3 V.

\subsection{Interfaz UART}



\section{Arquitectura de firmware}

Como se mencionó anteriormente, la arquitectura del firmware del proyecto tiene como base la utilización de FreeRTOS y los recursos que ofrece. Esto permite la implementación de un diseño eficaz y sencillo mediante la división en dos partes bien definidas: la parte de procesamiento de datos (o \textit{backend}) y la parte gráfica que interactúa con el usuario (o \textit{frontend}).

\subsection{Máquina de estados}

En la \ref{fig:diagFSM} se

%\begin{figure}[H]
%\centering
%\includegraphics[width=0.5\textwidth]{./Figures/diagFSM.jpg}
%\caption{Máquina de estados.}
%\label{fig:diagFSM}
%\end{figure}

\subsection{Capa de abstracción del amplificador}

Con el objetivo de generar una estructura jerárquica por capas, se decidió crear un \textit{wrapper} que contenga  

\subsection{Driver del ADC}

%\ref{tab:configADC}

%\begin{table}[H]
%	\centering
%	\caption{Configuración del ADC.}
%	\begin{tabular}{l c}
%		\toprule
%		\textbf{Característica} & \textbf{Configuración} \\
%		\midrule
%		Multímetro			&  UNI-T 	 \\
%		Fuente 				& XXX	     \\
%
%		\bottomrule
%		\hline
%	\end{tabular}
%	\label{tab:configADC}
%\end{table}

%\begin{figure}[H]
%\centering
%\includegraphics[width=0.5\textwidth]{./Figures/rutinaADC.jpg}
%\caption{Rutina de adquisición de valores analógicos.}
%\label{fig:rutinaADC}
%\end{figure}

\subsection{Driver de la pantalla LCD}

%\ref{tab:configSPI}

%\begin{table}[H]
%	\centering
%	\caption{Configuración de los SPI del LCD.}
%	\begin{tabular}{l c}
%		\toprule
%		\textbf{Característica} & \textbf{Configuración} \\
%		\midrule
%		Multímetro			&  UNI-T 	\\
%		Fuente 				& XXX	    \\
%
%		\bottomrule
%		\hline
%	\end{tabular}
%	\label{tab:configSPI}
%\end{table}